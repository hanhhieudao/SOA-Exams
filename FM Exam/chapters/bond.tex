\chapter{Bond}

\section{Bond}

\begin{definition}
    A \textbf{bond} is a type of debt instrument made by investors to a borrower (government/corporation). 
    The borrower promises to pay: interest (coupons) and principal (face value) at maturity. 
\end{definition}

\subsection{Key terms}

\begin{comments}
    \begin{itemize}
        \item \textbf{Term}: Time from issue to maturity.
        
        \item \textbf{Maturity Date}: Final payment date.
        \item \textbf{Yield}: Actual return for the investor, depending on the price.
    \end{itemize}
\end{comments}

\begin{formula} 
    Regarding to bond pricing formulas, key notations are as follow: 

    \begin{itemize}
        \item \textbf{Par Value / Face Value $F$}: Original issue price of the bond which does not change over time. 
        
        \item \textbf{Coupon Rate $r$}: Interest rate applied to the face value and is set by issuer (fixed). 
        
        \item \textbf{Price $P$}: What the investor pays for the bond.
        \item \textbf{Redemption Value of Bond $C$}: $ F = C $ until otherwise stated. $C$ is the amount the bondholder gets at maturity. If the bons is called early, $C$ = Call price. 
        \item \textbf{Interest rate per payment period $i$}: Fluctuates based on market. 
        \item \textbf{Number of coupon payments $n$}
    \end{itemize}

\end{formula}


%-----------------
\section{Book Value}
\begin{definition}
    \textbf{Book Value} of a bond is the price (=present value) of the bond at any time between the purchase and maturity date.
    Define $B_k$ is the book value (amortized value) of the bond immediately after the $k^{th}$ payment. 
\end{definition}

\begin{comments}
    Prospectively, the book value of a bond is given by: 
    \[
    B_k = \text{PV(Remaining Payments)} = Fr \cdot a_{\angl{n-k}i} + C \cdot v^{n-k}
    \]
    Retrospectively, the book value of a bond is given by: 
    \[
        B_k = \text{FV(P)} - \text{FV(Past Payments)} = P \cdot (1+i)^k - Fr \cdot s_{\angl{k}i}
    \]
\end{comments}




%---------------------------------
\subsection{Principal and Interest}
\begin{comments}
\textbf{Principal: }This is the original amount the bond issuer promises to pay back at maturity.
            It is the face value/par value and investor get this once, at the end of the bond's life.  \\
            \[
                \text{Principal = Redemption Value}
            \]
\textbf{Interest: } Interest each period is the coupon payment. 
\[
\text{Interest each period} = \text{Coupon rate x Face Value}
\]
\end{comments}





%---------------------
\subsection{Principal and Interest of Bond in Book Value}
\begin{comments}
    In book value calculations, we break coupon into 2 parts: 
    \begin{itemize}
        \item \textbf{(Expected) Interest: }Interest earned at yield rate = Yield rate x Book Value
        \item \textbf{Principal adjustment at the $k^{th}$ coupon }= Coupon - Interest 
    \end{itemize}
\end{comments}

\begin{formula}
    Interest earned at yield rate: 
    \[
        I_k = i \cdot B_{k-1}
    \]  
    Principal adjustment after the $k^{th}$ coupon payment: 
    \[
        P_k = Fr - I_k = B_{k-1} - B_k
    \]
    \begin{itemize}
        \item Premium: $P_k > 0$ Book value decreases 
        \item Discount: $P_k < 0$ Book value increases
    \end{itemize}
\end{formula}

\begin{formula}
    Total principal adjustment from time k to m: 
    \[
        \sum_{j=k+1}^{m} P_{k+1} = B_k - B_m
    \]
    Total interest earned from k to m: 
    \[
        \sum_{j=k+1}^{m} I_j = \sum C - \sum P_j = (m-k) \cdot C - (B_k - B_m)
    \]
\end{formula}





%-------------------------------------------
\section{Bond yield}

\begin{definition}
    \textbf{Bond Yield/Yield to Maturity (YTM)} is the rate of return $i$ you'll earn if you hold the bond to maturity, 
    assuming all coupons are reinvested at the same rate. It is expressed as percentage ($\%$). Components of YTM: 
    \begin{enumerate}
        \item Current bond price $P$
        \item Coupon payments $Fr$
        \item Time to maturity $n$
        \item Face Value/Par value $F$
    \end{enumerate}
\end{definition}

\begin{formula}
    \textbf{(Nominal Yield)} 
    \[
        \text{Nominal Yield} = \frac{\text{Annual Coupon}}{\text{Par Value}} = \frac{Fr}{F}
    \]  
    Nominal Yield is the \textbf{annualized} rate of return you earn based on the bond's Face value. 
    Unit of $r$ is \%/year. 
\end{formula}

\begin{formula}
    \textbf{(Current Yield)} 
    \[
        g = \text{Current Yield} = \frac{\text{Annual Coupon}}{\text{Bond Price}}
    \]  
    Current yield is the \textbf{annualized} rate you earn based on what you actually paid for the bond. 
\end{formula}

\begin{comments}
    \begin{itemize}
        \item $g > i$: Premium bond
        \item $g < i$: Discount bond
    \end{itemize}
\end{comments}

\begin{formula}
    You buy a bond at the current market price and holding it to maturity. \textbf{Yield to Maturity} 
    is the annualized rate of return you expect to have. In other words, it discounts the future cash flows
    to be equal to the market price of the bond. 
\end{formula}

%---------------------------------------------
\section{Bond pricing formula}

\begin{formula} (Bond Price)
    For a bond with coupons, its price is 
    \[
        P(i) = \text{PV of Coupon payments} + \text{PV of Redemption value} = F \cdot r \cdot a_{\angl{n}i} + Cv^{n}
    \]
\end{formula}



%----------------
\subsection{Base amount formula}
\begin{formula} {Base amount formula}
    \[
        \text{P} = \text{Base} + \text{Adjustment} = Fr + (C - Fr)v^{n}
    \]
\end{formula}







%-------------------------
\subsection{Bonds with Geometric Coupon Payments}


\begin{formula}
    If a bond has a geometrically varying coupons (i.e., the first coupon is X and 
    each subsequent coupon is (1+k) times the preceding one) with the yield rate $i$, 
    PV = Price of the bond is: 
    \begin{align}
        \text{PV} &= \text{PV}_{\text{Redemption value}} + \text{PV}_{\text{coupon annuity}} \\
                &= C \cdot v^{t} + X \cdot \frac{1-(\frac{1+k}{1+i})^{n}}{i-k}
    \end{align}
\end{formula}

%-------------------
\subsection{Bonds with Different Frequencies for Coupon and Yield rates}

\begin{center}
\renewcommand{\arraystretch}{1.5}
\begin{tabular}{@{} >{\raggedright\arraybackslash}p{3cm} >{\raggedright\arraybackslash}p{5cm} >{\raggedright\arraybackslash}p{7cm} @{}}
\textbf{Case} & \textbf{Formula} & \textbf{Interpretation} \\
\midrule
$n < k$ (Few coupons, many yield periods) & 
\( P = Fr \cdot a_{\overline{n}|}^{(k)} + C v^n \) & 
Coupon rate is compounded less frequently. Total payment periods is $nk$.\\
\midrule
$n > k$ (Frequent coupons, fewer yield periods) & 
\( P = \frac{Fr \cdot a_{\overline{n}|}}{s_k} + C v^n \) & 
Coupons occur more frequently. Total payment periods is $n/k$ \\
\midrule
$n = k$ & \( P = Fr \cdot a_{\overline{n}} + C v^n \) & Payments match discounting intervals. \\
\bottomrule
\end{tabular}
\end{center}










%----------------------------------
\section{Premium/Discount}

\begin{comments}
    Bonds dont always trade at \textbf{par value} (i.e. $P$ is not always equal to $F$). This is because
    coupon rate $r$ is different from (market) yield rate/YTM $i$. In addition, price of bond $P$ is 
    dependent on market yield $i$. 
    \begin{itemize}
        \item If $r = i$ or $Fr = Ci$, bond is sold at par. 
        \item If $r > i$ or $Fr > Ci$, bond is sold at a premium. 
        \item If $r < i$ or $Fr < Ci$, bond is issed at a discount. 
    \end{itemize}
    Note that $F = C$ by default. 
\end{comments}


\begin{comments}
    If $r < i$, the bond is less attractive to investors. So the issuer must lower the price to make it appealing and the bond is sold at discount. 
    Similarly, if $r > i$, the issuer will raise the price as investors are willing to pay higher price for the bond.

    \begin{tabular}{|c|c|c|}
    \hline
    \textbf{Condition} & \textbf{Price Compared to \( C \)} & \textbf{Called} \\
    \hline
    \( r < i \) & \( P < C(=F) \) & \textbf{Discount bond} \\
    \( r = i \) & \( P = C(=F) \) & \textbf{Par} \\
    \( r > i \) & \( P > C(=F) \) & \textbf{Premium bond} \\
    \hline
    \end{tabular}

\end{comments}

\begin{formula}
    Price of a bond can be calculated using Premium/Discount: 
    \[
        P(i) = C + (F \cdot r - C \cdot i) \text{a}_{\angl{n}i}
    \]
    where $Fr - Ci$ is the premium or discount $\textbf{per period}$. They are cash flows over $n$ periods and 
    seen as an annuity of coupon payments. 
    \begin{itemize}
        \item \textbf{Premium}: $P-C = (Fr-Ci)a_{\angl{n}i}$
        \item \textbf{Discount}: $C-P = (Ci-Fr)a_{\angl{n}i}$
    \end{itemize}

    Note that $Fr$ is the coupon payment, and $Ci$ is the market interest payment. 
\end{formula}









%---------------------------------------------
\section{Bond duration}
\subsection{Par Bond Duration}

\begin{comments}
    When the bond is sold at par (i.e., when the yield $i=r$), its Macaulay Duration is: 
\begin{align*}
\text{MacD} &= \frac{F(r(Ia)_{\angl{n}} + nv^n)}{Fra_{\angl{n}} + Fv^{n}} \\
            &= \frac{r(Ia)_{\angl{n}} + nv^n}{ra_{\angl{n}} + v^{n}}  \\
            &= \frac{r(1+i)a_{\overline{n}|i} + (i - r)n v^n}{r + (i - r)v^n}
\end{align*}

Since $i=r$ (bond sold at par), the equation becomes
\[
    \text{MacD} = \frac{r(1+i)a_{\angl{n}i}}{r} = (1+i)a_{\angl{n}i} = \ddot{a}_{\angl{n}i}
\]
\end{comments}
\begin{formula}
    The Macaulay Duration of a par bond is 
    \[
    \text{MacD} = \ddot{a}_{\angl{n}i}
    \]
\end{formula}




%-----------------
\section{Callable Bond}
\begin{definition}
    Bonds that can be redeemed by the issuer before maturity. T
    he issuer decides when to call, often to save on interest costs.
\end{definition}

\begin{comments}
    Callable bonds can be redeemed early by the issuer. The issuer decides when to call (redeem) the bond, but you, 
    the investor, don’t get to choose. So when you're trying to figure out how much to pay for a callable bond, you must assume the issuer 
    will act in their own best interest, not yours.

\end{comments}

\begin{tabular}{>{\bfseries}l p{3cm} p{3.5cm} p{4cm}}
\toprule
\textbf{Case} & \textbf{What's Happening} & \textbf{What You Assume as Investor} & \textbf{Why?} \\
\midrule
\textbf{Premium Bond} & Coupon rate $>$ yield rate & Bond is called \textbf{early} & Issuer wants to stop overpaying and call the bond at the first possible date\\
\textbf{Discount Bond} & Coupon rate $<$ yield rate & Bond is called \textbf{late} & Issuer benefits from paying low coupons longer and hold the bond until the final maturity date \\
\bottomrule
\end{tabular}


\begin{comments}
When you're buying a callable bond, you already know:
\begin{itemize}
    \item Call price = Redemption value 
    \item Call schedule (when it can be called)
    \item Coupon rate 
    \item Market yield you want (your target of return)
\end{itemize}
So you can \textbf{adjust the purchase price you're willing to pay}. 
\end{comments}


\subsection{Endpoints shortcut}

\begin{comments}
    For callable bond, all possible call dates are evaluated when investor consier the purchase price. 
    You only need to eavluate the bond price at the endpoints of the time interval in which 
    the bond is possibly called. 
\end{comments}

\begin{formula}
    \textbf{(Lowest yield - worst case)} at the endpoints:  You get lowest yield and this is the worst vase
    when the callable bond is called early" 
    \begin{itemize}
        \item Earliest call on left-endpoint - Premium bond
        \item Earliest call on right-endpoint - Discount bond
    \end{itemize}

\end{formula}
\chapter{Immunization}

\section{Assets and liabilities}

\begin{comments}
Cash flows: 
    \begin{itemize}
        \item Asset inflows: $A_0, A_1, ... , A_n$
        \item liability outflows: $L_0, L_1, ... , L_n$
        \item At each time $t$, $R_t = A_t - L_t$ = Net cash flow at time $t$
    \end{itemize} 
\end{comments}


















\section{Redington Immunization}

\begin{definition}
    Redington immunization is a strategy to \textbf{protect a portfolio} (assets vs liabilities) 
    from small changes in interest rates. 
\end{definition}


\begin{comments}

    Let $P(i)$ be the present value of all the net cash flows at interest rate $i$. Since we 
    want the value of portfolio to \textbf{not drop} when $i$ changes slightly, that means 
    $P(i)$ should be at a minimum at the current target rate $i_0$. 
\begin{table}[htbp]
\centering
\caption{3 Conditions for Immunization}
\begin{tabular}{lll}
\toprule
\textbf{Condition} & \textbf{Meaning} & \textbf{Purpose} \\
\midrule
$P(i) = 0$ & PV of assets = PV of liabilities & Start balanced \\
$P'(i) = 0$ & Modified durations match & No change for small $\Delta i$ \\
$P''(i) > 0$ & Positive convexity & Changes in rate increase value \\
\bottomrule
\end{tabular}
\end{table}


\end{comments}



%------------------------
\section{Full Immunization}

\begin{definition}
    (Full immunization) Full immunization protects a portfolio from \textbf{any} interest rate changes, 
    not just small changes. 
\end{definition}

\begin{formula}
    Full Immunization Conditions at $i = i_0$: 
    \begin{enumerate}
        \item $PV_{A}(i_0) = PV_{L}(i_0)$ or PV of assets equals to PV of liabilities
        \item $PV_{A}'(i_0) = PV_{L}'(i_0)$ or $\text{ModD}_{A}(i_0) = \text{ModD}_{L}(i_0)$
        \item $PV_{A}''(i_0) \> PV_{L}''(i_0)$ or $\text{ModC}_{A}(i_0) = \text{ModC}_{L}(i_0)$
    \end{enumerate}
\end{formula}

\begin{comments}
    The 3rd condition is the \textbf{timing condition}, which means that there has to be 
    asset cash flow \textbf{before and after} each liability cash flow. It helps to reduce 
    interest risk: no matter how interest rates move, your total asset value will always be 
    enough to cover the liability. 
\end{comments}
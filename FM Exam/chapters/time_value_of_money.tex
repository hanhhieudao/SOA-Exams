\chapter{Interest Theory}
%-------------------------------------------------------
\section{Amount and Accumulation Functions}

\begin{definition}
    \textbf{Amount function} $A(t)$ refers to value of the investment at time t. 
\end{definition}

\begin{definition}
    \textbf{Accumulation function} $a(t)$ refers to value of \$1, which was invested at time 0, at time t. 
\end{definition}

\begin{formula}
    relationship between A(t) and a(t) is 
    \[
        A(t) = A(0) \cdot a(t)
    \]  
    where 
    \begin{itemize}
        \item A(0) is money you invested at time 0. 
        \item a(t)  tells how much \textbf{each dollar} grows to by time t. 
    \end{itemize}
\end{formula}



\begin{itemize}[leftmargin=*,nosep]
    \item \textbf{Simple Interest}
    \begin{itemize}[leftmargin=*]
        \item Accumulation function: $a(t) = 1 + it$
        \item Growth type: Linear
    \end{itemize}

    \item \textbf{Compound Interest}
    \begin{itemize}[leftmargin=*]
        \item Accumulation function: $a(t) = (1 + i)^t$
        \item Growth type: Exponential
    \end{itemize}

    \item \textbf{Continuous (Force of Interest)}
    \begin{itemize}[leftmargin=*]
        \item Accumulation function: $a(t) = e^{\delta t}$
        \item Growth type: Exponential (Smoothest)
    \end{itemize}
\end{itemize}








%-------------------------------------------------------
\section{Force of Interest}

\begin{definition}
    The \textbf{force of interest}, denoted by  $\delta$, measures how fast money grows in any specific time.
\end{definition}

\subsection{Constant Force of Interest}

\begin{definition}
    Start with a nominal compound interest $i^{(m)}$ converted $m$ times per period. Let $m$ go to infinity, 
    money is compounded \textbf{continuously}: 
    \[
            \delta(t) = \lim_{m \to \infty} i^{(m)}
    \]
    When using \textbf{compound interest}, $\delta_t$ is constant. 
\end{definition}

\begin{formula}
    The initial amount of money grows exponentially with a constant rate $\delta$. 
    \[
        A(t)  = A(0) \cdot a(t) = A(0) \cdot e^{\delta t}
    \]

    where accumulation function $a(t) = e^{\delta t}$.
\end{formula}

\begin{formula}
    To convert annual effective rate to \textbf{constant Force of Interest}, and conversely: 
    \[
        \delta = \lim_{m \to \infty} i^{(m)} = \ln(1+i)
    \]

        \[
        i = e^{\delta} - 1   
    \]

    \begin{itemize}
        \item $i$ = effective annual interest rate
        \item $i^{(m)}$ = nominal rate compounded m times a year
    \end{itemize}
    
\end{formula}



\begin{formula}
    When $\delta$ is constant, then $\delta(t)$ = $\delta$ for all $t$. \\
    \[
        \text{FV} = A(t) = A(0) \cdot e^{\delta t}
    \]

    \[
        \text{PV} = A(t) \cdot e^{-\delta t}
    \]
    where accumulation function $a(t) = e^{\delta t} = (1+i)^t$. 
    
\end{formula}



%-------------------------------
\subsection{Varying Force of Interest}
\begin{definition}
    Force of Interest which \textbf{changes over time} is a variable and a function of time $\delta(t)=\delta_t$. 
    \[
        \delta_t = \frac{a'(t)}{a(t)} = \frac{A'(t)}{A(t)}
    \]

    Given $\delta_t$, we can recover accumulation function $a(t)$. 
    \[
        a(t) = e^{\int_{0}^{t}\delta_t dr}
    \]  
\end{definition}

\begin{formula}
    (Present value and Future value)
    \[
    \text{FV} = A(t) = A(0) \cdot e^{\int_{0}^{t}\delta_t dr}
    \]
    \[
    \text{PV} = A(0) = A(t) \cdot e^{-\int_{0}^{t}\delta_t dr}
    \]

\end{formula}

\begin{comments}
    The amount of interest earned over $n$ periods is 
    \[
    A(n) - A(0) = I_1 + I_1 + \dots + I_n = \int_{0}^{n}A'(t)dt = \int_{0}^{n}A(t) \delta_t dt
    \]
\end{comments}



%-------------------------------
\section{Present value}

\subsection{Accumulation function with compound and simple interests}
\begin{definition}
    With \textbf{compound interest}, the accumulation function is 
    \[
        a(t) = (1+i)^t
    \]  
    Reminds that $a(t)$ tells how \$1 grows over t periods at (compound) interest rate $i$. 
\end{definition}

\begin{definition}
    With \textbf{simple interest}, the accumulation function is 
    \[
        a(t) = 1+it
    \]  
    Reminds that $a(t)$ tells how \$1 grows over t periods at (simple) interest rate $i$. 
\end{definition}



\subsection{Discounting}

\begin{comments}
    \begin{enumerate}
        \item \textbf{Discounting} What is \$1 in future worth today? $\rightarrow (1+i)^t$ after t periods
        \item \textbf{Accumulation} What does \$1 today grow to in future? $\rightarrow \frac{1}{(1+i)^t}$ after t periods
    \end{enumerate}
\end{comments}

\subsection{Discount factor}

\begin{definition}
    \textbf{Discount factor} converts future money into present value. For $t$ periods, 
    the discount factor is 
    \[
        v^t = \frac{1}{(1+i)^t}
    \]  

    Discount factor during nth period is

    \[
    (1+i_n)^{-1} = \frac{A(n-1)}{A(n)}
    \]
\end{definition}


\subsection{Present Value}

\begin{definition}
    (\textbf{Lump Sum}) Present Value of an Lump sum amount with compound interest: 
    \[
        \text{PV} = \frac{\text{FV}}{(1+i)^{t}} = \text{FV} \cdot v^t = \text{FV} \cdot \ \frac{1}{a(t)}
    \]  
\end{definition}
   

\begin{definition}
    (\textbf{Cash Flow}) Present Value of a cash flow: 
    \[
        \text{PV}_{t=0} = \sum_{k=1}^{n} C_k (a(t_k))^{-1}
    \]
\end{definition}

%---------------------------
\section{Future Value}
\begin{definition}
    With cash flows $C_1, C_2, \dots, C_n$ received at times $t_1, t_2, \dots, t_n$, then
    \[
        \text{FV}_{t=n} = \sum_{k=1}^{n} C_k a(t_k)
    \]
    \begin{itemize}
        \item $C_k$: cash flow received at time $t_k$
        \item $a(t_k)$: accumulation factor from time $t_k$ to the final time $t=n$
        \item $\text{FV}_{t=n}$: future value at time $t=n$
    \end{itemize}
\end{definition}

\begin{formula}
    (\textbf{Accumulation factor and Deposits at Different time})  of \$k is deposited
    at time $s$, and you want to know its value at time $t>s$, use
    \[
        \text{Future Value} = k \cdot \frac{a(t)}{a(s)}
    \]
    where $\frac{a(t)}{a(s)}$ is an accumulation factor. 
\end{formula}


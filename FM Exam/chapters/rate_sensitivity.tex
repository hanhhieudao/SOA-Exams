\chapter{Measures of Interest Rate Sensitivity}



\section{Inflation}
\begin{definition}
    Inflation = general rise in prices of goods and services over time. It reduces purchasing power of money.
\end{definition}

\begin{comments}
    Inflation and Interest Rates: inflation and interest rates move together over time. Investors 
    demand higher interest to compensate for future inflation. 
\end{comments}

\begin{formula}
    Let $\pi$ is the inflation rate. 
    \[ 1 + i_{\text{real}} = \frac{1 + i_{\text{nominal}}}{1 + \pi} \]
\end{formula}

%---------------------------
\subsection{Payments grow with inflation}
\subsubsection{Present Value}
\begin{definition}
    Each future payment grows by constant \textbf{inflation rate}. 
\end{definition}

\begin{formula} (PV iwth adjusted payments)
    \[
PV = R\left[\frac{1+r}{1+i} + \left(\frac{1+r}{1+i}\right)^2 + \cdots + \left(\frac{1+r}{1+i}\right)^n\right]
    = R(1+r) \cdot \frac{1 - \left(\frac{1+r}{1+i}\right)^n}{i - r}
\]
We discount the annuity with \textbf{nominal rate} as payments have been adjusted with inflation rate.
\end{formula}
\begin{formula} (PV with ray payments, while inflation exists)
    \[
    PV = R\left[\frac{1}{1+i_0} + \frac{1}{(1+i_0)^2} + \cdots + \frac{1}{(1+i_0)^n}\right] = R \cdot a_{\overline{n}|i_0}
    \]

We discount the annuity with \textbf{real interest rate.}
\end{formula}
    

\subsection{Summary}
\begin{center}
\begin{tabular}{|l|l|l|}
\toprule
\textbf{Scenario} & \textbf{Formula} & \textbf{When to Use} \\
\midrule
Payments grow with inflation & 
$PV = R(1+r) \cdot \frac{1 - \left(\frac{1+r}{1+i}\right)^n}{i - r}$ & 
Use nominal rate $i$ \\
\midrule
Payments fixed in real terms & 
$PV = R \cdot a_{\overline{n}|i_0}$ & 
Use real rate $i_0$ \\
\bottomrule
\end{tabular}
\end{center}

\subsubsection{Accumulated Value}
\begin{formula} (AV in nominal dollars (not adjusted for inflation))
    \[ \text{AV} = P(1+i_\text{nominal})^n \]
    This is the raw future value of your investment.
\end{formula}

\begin{formula} (AV adjusted for inflation)
    \[ \text{AV} = P (\frac{1+i}{1+r})^n = P(1+i_\text{real}) \]
    This refects the true purchasing power of your money.
\end{formula}

%----------------------------
\section{The Term structure of Interest Rates and Yield Curves}
\subsection{Term}
\begin{definition}
    \textbf{Term}: The length of time until an investment/loan matures/ends. It is the duration
until you get your money back.
\end{definition}







\subsection{Spot rate}
\begin{definition}
    \textbf{Spot rate} is the yield to maturity/(single rate of annual return) of a zero-coupon bond/(no cash flows).  
Spot rate is always based on time zero. It is a rate for one-time future payment.
    \[ v_t = \frac{1}{(1+s_t)^t} \]

    \begin{itemize}
        \item $v_t$: discount factor
        \item $s_t$: spot rate
    \end{itemize}
\end{definition}

\begin{comments}
    Generally — the longer the investment term, the higher the interest rate, because 
    \begin{itemize}
        \item More time = more risk (like inflation, uncertainty, default). 
        \item Investors want extra return for locking money up longer, hence they charge higher rates for longer loans.
    \end{itemize}

    Key differences between: \textbf{Zero-coupon bond with spot/forward rates} and 
    \textbf{Annuity using varying spot rates} 

    \begin{tabular}{@{} l l l @{}}
\toprule
\textbf{Concept} & \textbf{Zero-Coupon Bond} & \textbf{Annuity with Varying Rates} \\
\midrule
Cash Flows & Single payment at end & Multiple payments/cash flows each year \\
Discounting Method & Compound using forward rates & Discount each payment with spot rates \\
Formula Used & 
\( (1 + s_n)^n = \prod_{i=0}^{n} (1 + f_{[i, i+1]}) \) & 
\( PV = \sum \frac{C_t}{(1 + s_t)^t} \) \\
Use Case & Zero-coupon bond pricing / yield & Valuing pension plans, loans, etc. \\
\bottomrule
\end{tabular}

\end{comments}

\subsection{Yield}
\begin{comments}
    Extend the table to a continous graph, where y-axis is \textbf{yield} (interest rates/spot rates
    of risk-free bonds) and x-axis is \textbf{maturity}, we obtain a yield curve. Yield curve can be 
    upward-sloping (rates expected to rise), flat (all terms have same rate), and inverted (short-term > long-term, 
    a signal of recession).  

\end{comments}

\begin{table}[h]
\centering
\begin{tabular}{lc}
\toprule
\textbf{Length of investment (years)} & \textbf{Interest rate (Spot rate)} \\
\midrule
1 year & 3\% \\
2 year & 4\% \\
3 year & 6\% \\
4 year & 7\% \\
\bottomrule
\end{tabular}
\end{table}

\begin{definition}
    \textbf{Yield Curve} is a graph of \textbf{spot rates} versus maturity time. 
\end{definition}


\begin{comments}
    
\end{comments}

\begin{definition}
    \textbf{Yield to Maturity: } A \textbf{single average rate} that discounts all cash flows of a bond. 
    \begin{itemize}
        \item The bond is held to maturity. 
        \item The bond does not default. 
        \item Reinvestment of the bond and all coupons is executed at the original YTM. 
    \end{itemize}
\end{definition}
\begin{formula}
    When spot rates $i_t$ vary by year, NPV is 
    \[ \text{NPV} = \sum_{t=0}^{n} \frac{c_t}{(1+i_t)^t}\]
\end{formula}




\subsection{Forward Rate}
\begin{definition}
    \textbf{Forward rate:} the \textbf{interest rate} agreed on today for borrowing or investing money in 
    the future from time n to time n+m. It tells what the market expects interest rates to be. 

    \[ v_{[n, m-n]} = \frac{1}{(1+f_{[n, n+m]})^{m-n}}\]
\end{definition}

\begin{formula} (Connect Spot rate with Forward rates)
    \begin{itemize}
        \item Spot Rate (s): Set by the current market — changes daily with supply/demand. 
        \item Forward Rate (f): Calculated from spot rates. 
    \end{itemize}

    \[
        (1+s_n)^n \dot (1+f_{[n, n+m]})^m = (1+s_{n+m})^{n+m}
    \]
    \[
        (1 + s_{[0,n]})^n = (1 + f_{[0,1]}) \cdot (1 + f_{[1,2]}) \dots (1 + f_{[n-1,n]})
    \]
\end{formula}


%--------------------------------------------------
\section{Macaulay and Modified Durations}
\begin{comments}
    Why \textbf{Duration} matters? \\
    - Duration measures the sensitivity of a bond's price to changes in interest rates. \\
    - Duration reflects the timing of cash flowe (i.e. when you'll get your money back). \\
\end{comments}

\begin{comments}
\begin{tabular}{@{}lll@{}}
\toprule
\textbf{Types of Duration} & \textbf{Definition} & \textbf{Formula} \\
\midrule
Term to Maturity & Time until final payment (not very useful with coupons). & --- \\[1ex]
Equated Time & Weighted average of payment times (weights = cash flows). &

$\bar{t} = \frac{\sum t R_t}{\sum R_t}$ \\
Macaulay Duration & Weighted average of present values of payments. &
$d = \frac{\sum t \nu^t R_t}{\sum \nu^t R_t}$ \\
Modified Duration & Measures price sensitivity to interest changes. &
$\text{ModDur} = \frac{d}{1 + i}$ \\
\bottomrule
\end{tabular}
\end{comments}

\subsection{Average term-to-maturity}

\begin{definition}
    Average Term-to-Maturity: "On average, when do I receive my money back?"
\end{definition}

\begin{comments}
    Setup: For zero-coupon bond, there is only one payment (at maturity). It means shorter maturity
    has faster cash back. But for coupon bons with multiple cash flows over time, term-to-maturity
    ignores earlier coupon payments. Let's say you're paid \$100 in year 1 (coupon), \$200 in year 2 (coupon), and \$700 (final coupon + principal) in year 3. 
    The average term-to-maturity might be around 2.6 years - it tells on average, you get back total \$1000, not just 
    \$700 in year 3. 
\end{comments}

\begin{formula}

If a bond pays cash flows \( C_0, C_1, \dots, C_n \) at times \( t_0, t_1, \dots, t_n \), then the \textbf{Equated Time} is given by:

\[
\text{Equated Time} = \frac{\sum_{i=0}^{n} C_i \cdot t_i}{\sum_{i=0}^{n} C_i}
\]

It tells how quickly your investment is returned, on average. 
\end{formula}



%--------------------------------
\subsection{Macaulay Duration}
\begin{definition}
    It improves the upon equated time by using \textbf{present values} instead of 
    just raw cash flows. Each cash flow is now discounted to present, thus the weighted 
    average time of cash flows is more precise.
\end{definition}

\begin{formula}
The \textbf{Macaulay Duration} \( MacD \) is given by:

\[
MacD(i) = \frac{\sum_{t=0}^{n} t \cdot \nu^t R_t}{\sum_{t=0}^{n} \nu^t R_t}
\]

\begin{itemize}
    \item \( R_t \) be the cash flow at time \( t \) 
    \item \( \nu^t = \frac{1}{(1 + i)^t} \) is the discount factor at time \( t \)
    \item \( i \) is the effective rate of interest per period.
\end{itemize}

Note that duration depends on $i$. When $i$ = 0, the MacD is equal to the equated time formula. 
When there is only 1 future payment, duration is equal to time of payment (MadD = t). 

\end{formula}

\subsection{Macaulay Duration and Cauchy-Schwarz Inequality}
\begin{comments}

Recall the Macaulay duration:
\[
d = \frac{\sum_{t=0}^{n} t \cdot \nu^t R_t}{\sum_{t=0}^{n} \nu^t R_t}, \quad \text{where } \nu^t = \frac{1}{(1+i)^t}
\]

By Cauchy-Schwarz inequality, we see that

\[
\frac{d}{di} MacD(i) < 0
\]

So, Macaulay duration decreases as the interest rate increases.
\end{comments}











%--------------------------------

\section{Modified Duration (Volatility)}

\begin{definition} (Modified Duration)
    Measures how sensitive a \textbf{bond’s price} is to a small change in its \textbf{yield-to-maturity (YTM)}. 
    As yield rises, price falls, and vice versa - inverse relationship. 
\end{definition}

\begin{comments}
    Setup: \\

    The price of a bond is the present value of its cash flows:
    \[
            P(i) = \sum_{t=0}^{n} \frac{C_t}{(1+i)^t}
    \]  

    Take the derivative of P(i) w.t. $i$ (yield rate). This is the \textbf{rate of change}
    of bond price when $i$ changes: 

    \[
            \frac{dP(i)}{di}
    \]

    We want to express the percentage change in price of bond by dividing the derivative 
    by the price, with the minus sign as price drops when $i$ increases. We obtained \textbf{volatility}
    which tells how sensitive the PV of bond's price to interest rate changes. 
    \[
        \text{Volatility} = - \frac{1}{P(i)} \cdot \frac{dP(i)}{di} = -\frac{P'(i)}{P(i)}
    \]

    Real world meaning: if Volatility = 5, then if $i$ increases by 1\% bond price drops about 5\%. 
    It's a linear approximation of the price-yield curve i.e. the \% change of $P(i)$. \\

    The standard derivative identity is: 
    \[
        \frac{d}{di}[lnP(i)] = \frac{P'(i)}{P(i)}
    \]

    Thus, volatility, denoted by $\bar{v}$, becomes: 
    \[
        \bar{v} = - \frac{d}{di}[lnP(i)] = - \frac{P'(i)}{P(i)}
    \]

    Volatility is often called \textbf{modified duration}. Now we derive $P(i)$ by taking
    the derivative with respect to $i$ : 

    \[
    P'(i) = \frac{d}{di} \left[ \sum_{t=0}^n \frac{R_t}{(1+i)^t} \right] = - \sum_{t=0}^n t(1+i)^{-t-1} R_t = - \sum_{t=1}^{n} \frac{t \cdot R_t}{(1+i)^{t+1}}
    \]

    Then plug into the volatility formula:

    \[
    \bar{v} = -\frac{P'(i)}{P(i)} = \frac{\sum_{t=0}^{n} \frac{t \cdot R_t}{(1+i)^{t+1}}}{\sum_{t=0}^{n} \frac{R_t}{(1+i)^{t}}} = \frac{\sum_{t=0}^{n} t \cdot v^{t+1} \cdot R_t} {\sum_{t=0}^{n} t \cdot v^{t} \cdot R_t}
    \]

    Express $\bar{v}$ in terms of MacD: 
    \[ \bar{v} =  \frac{\sum_{t=0}^{n} t \cdot v^{t+1} \cdot R_t} {\sum_{t=0}^{n} t \cdot v^{t} \cdot R_t} 
    = v \cdot  \frac{\sum_{t=0}^{n} t \cdot v^{t} \cdot R_t} {\sum_{t=0}^{n} t \cdot v^{t} \cdot R_t} = \text{MacD} \cdot v = \frac{\text{MacD}}{1+i}
    \]
\end{comments}
    

\begin{formula}
    \[ \text{Modified Duration} = \text{Macaulay Duration} \cdot v = \frac{\sum_{t=0}^{n} t \cdot v^{t+1} \cdot R_t} {\sum_{t=0}^{n} t \cdot v^{t} \cdot R_t}
    \]

    Remark: we assume that cash flows (payments) are fixed - they do not change with the interest rate changes.
\end{formula}


\section{MacaulayD vs. ModifiedD}


\begin{table}
\centering
\caption{Comparison of Macaulay Duration and Modified Duration}
\begin{tabular}{p{2.5cm}p{5cm}p{5cm}}
\toprule
\textbf{Feature} & \textbf{Macaulay Duration} & \textbf{Modified Duration} \\
\midrule
Definition & Weighted average time until all payments in a series are made & Sensitivity of bond price to interest rate changes \\

Formula & 
$\text{MacD} = -\frac{P'(\delta)}{P(\delta)}$ & 
$\text{ModD} = -\frac{P'(i)}{P(i)} = \text{MacD} \cdot (1+i) $ \\[10pt]

Units & Time (usually in years) & Percentage change per 1\% interest rate change \\

Interpretation & ``When'' you get your money back (on average) & ``How much'' the price changes when interest changes \\

Rate sensitivity? & Indirectly & Directly \\

Dependence on Interest Rate & No (once cash flows are fixed) & Yes (through denominator $1 + i$) \\
\bottomrule
\end{tabular}
\end{table}


\begin{comments}
    Interpretation: 
    \begin{itemize}
        \item $\frac{P'(i)}{P(i)}$ = change in PV per unit change in $i$. (in percentage)
        \item $\frac{P'(\delta)}{P(\delta)}$ = cheng in PV per unit change in $\delta$. (in time)
    \end{itemize}
\end{comments}
%----------------------------------

\section{Passage of Time}
\begin{definition}
    As time passes, the cash flows are getting closer. So naturally, the duration decreases. 
\end{definition}

\subsection{Macaulay Duration} (Passage of Time)
\begin{formula}
    \[
        \text{MacD}_{new} = \text{MacD}_{old} - (t_1 - t_0)
    \]
\end{formula}

MacD changes over time, as cash flows are getting closer. The difference between these 
two $\text{MacD}_{old}$ and $\text{MacD}_{new}$ is just the time has passed $(t_1) - t_0$. 


\subsection{Modified Duration} (Passage of Time)

\begin{comments}
    Convert $\text{ModD}_{\text{new}} = \text{MacD}_{\text{new}} \cdot v = [\text{MacD}_{old} - (t_1 - t_0)] \cdot v = \text{ModD}_{\text{old}} - v(t_1 - t_0)$
\end{comments}
\begin{formula}
    \[
    \text{ModD}_{\text{new}} = \text{ModD}_{\text{old}} - v(t_1 - t_0)
    \]
\end{formula}



%--------------------------------
\section{Convexity}

\begin{comments}
    \begin{itemize}
        \item  \textbf{Duration} gives a linear approximation of how bond price changes with interest rates (yield). 
        \item  \textbf{Convexity} gives a curvature of how bond price (non-linear) changes with interest rates (yield). 
    \end{itemize}
    In other words, convexity measures the rate of change of the volatility with respect to interest changes. 
    High convexity bonds: Lose less when yields go up, and gain more when yields go down.
\end{comments}

\begin{comments}
    Factors that increase convexity: 
    \begin{itemize}
        \item Maturity
        \item Coupon rate
        \item YTM
        \item Cash flow spread
    \end{itemize}
\end{comments}

\pgfplotsset{compat=1.18}

    \begin{tikzpicture}
\begin{axis}[%
    axis lines = center,
    xlabel = {Yield},
    ylabel = {Bond Price},
    label style = {anchor=north east},
    xmin = 0, xmax=5,     ymin=0, ymax=1,
    ticks = none,
    enlargelimits=false,
    clip=false,
%
    domain = 1:4,
    samples = 50,
    no marks
            ]
\addplot    {1/(1.75*x)};

\end{axis}
    \end{tikzpicture}



\subsection{Macaulay Convexity}

\begin{definition}
    \textbf{Macaulay convexity} is the weighted average of the squares of the time $t^2$, using 
    present values as weights. 

    \[
        \text{MacC} = \frac{P''(\delta)}{P(\delta)} = \sum_{t \geq 0}^{\infty} \left( \frac{C_t(1 + i)^{-t}}{P(i)} \right) t^2 = \frac{\sum_{t=0}^{n} t^2 \cdot v^t \cdot \text{CF}_{t}}{\sum_{t=0}^{n} v^t \cdot \text{CF}_{t}}
    \]

\end{definition}



\subsection{Modified Convexity}

\begin{formula} (Modified Convexity)
    \[ \text{Convexity} = \frac{P''(i)}{P(i)} = \frac{\sum_{t=0}^{n} t \cdot (t+1) \cdot v^{t+2} \cdot R_t} {\sum_{t=0}^{n} v^{t} \cdot R_t}\] 

    where 
    \begin{itemize}
        \item $P(i)$ is the PV of net cash flows at interest $i$
        \item $P''(i)$ tells how fast duration itself changes (i.e., rate of curvature)
    \end{itemize}
\end{formula}

    





%---------------------------------

\section{Approximation of Bond Price}
\begin{comments}
    We want to approximate how the bond price changes when the interest rate changes slightly from 
$i$ to $\Delta i $. 
\[
    P(i + \Delta i) \approx P(i) + \Delta i \cdot P'(i)
\]

To get percentage change, divide both sides by P(i): 
\[
\frac{P(i + \Delta i)}{P(i)} \approx 1 + \Delta i \cdot \frac{P'(i)}{P(i)} = 1 - \Delta i \cdot \text{ModD}
\]
The approximation is: 
\[ P(i + \Delta i) \approx P(i) \cdot [1 - \text{ModD} \cdot \Delta i ]\]
\end{comments}

\begin{formula} (1st-order Modified Approximation)
    \[
        P(i_n) \approx P(i_0) \cdot [1 - (i_n - i_0)(\text{ModD})]
    \]

    \[
        \Delta{P} = -\Delta{i} \cdot \text{ModD}
    \]
\end{formula}

\begin{comments}

    Now, we want the approximation in terms of MacD. Start with bond price as the sum of discounted cash flows: 
    \[
        P(i) = \sum_{t=0}^{n} \frac{C_t}{(1+i)^t}
    \]

    We approximate the bond price as a single lump-sum (K) as total amount of cash flows at average time which 
    is MacD: 
    \[
        P(i) \approx \frac{K}{(1+i)^{\text{MacD}}}
    \]

    When $i$ changes, $P(i)$ also changes. We take the ratio of new price to old price: 
    \[ 
        \frac{P(i_new)}{P(i_old)} \approx \frac{(1+i_{old})^{\text{MacD}}}{(1+i_{new})^{\text{MacD}}} = (\frac{1+i_{old}}{1+i_{new}})^{\text{MacD}}
    \]

\end{comments}

\begin{formula} (1st-order Macaulay Approximation)
    \[
        P(i_{\text{new}}) \approx P(i_{\text{old}}) \cdot (\frac{1+i_{old}}{1+i_{new}})^{\text{MacD}}
    \]
\end{formula}


%----------------------
\section{Bond Duration}

\begin{definition}
    \textbf{Bond duration} is a measure of the bond's Sensitivity to interest changes. 
\end{definition}

\begin{formula}
    For a bond of n annual coupons, face amount F, coupon rate r, and annual yield rate $i$: 
    \begin{itemize}
        \item Annuity-immediate: F $\cdot$ r
        \item Redemption value at maturity date t = n: F = C 
    \end{itemize}

    \[
        \text{MacD} = \frac{\sum_{t=1}^{n} t \cdot PV(\text{CF}_t)}{P}
    \]

    where 
    \begin{itemize}
        \item $PV(CF_{t})$ is the PV of the cash flow at time $t$ (coupon or principal)
        \item $P$ is the total PV (price) of the bond
    \end{itemize}
\end{formula}











%----------------------
\section{Note}
\begin{enumerate}
    \item Bond
        \begin{itemize}
            \item Buying a bond = outflow.
            \item Coupon payments = inflow.
            \item Maturity value = inflow.
        \end{itemize}
    \item Annuity
        \begin{itemize}
            \item Saving: regular payments = outflow. 
        \end{itemize}
    \item Loan
        \begin{itemize}
            \item Taking the loan = inflow.
            \item Loan repayments = outflow.
        \end{itemize}
\end{enumerate}




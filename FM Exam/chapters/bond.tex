\chapter{Bond}

\section{Bond}

\begin{definition}
    A \textbf{bond} is a type of debt instrument made by investors to a borrower (government/corporation). 
    The borrower promises to pay: interest (coupons) and principal (face value) at maturity. 
\end{definition}

\subsection{Key terms}

\begin{comments}
    \begin{itemize}
        \item \textbf{Term}: Time from issue to maturity.
        
        \item \textbf{Maturity Date}: Final payment date.
        \item \textbf{Yield}: Actual return for the investor, depending on the price.
    \end{itemize}
\end{comments}

\begin{formula} 
    Regarding to bond pricing formulas, key notations are as follow: 

    \begin{itemize}
        \item \textbf{Par Value / Face Value $F$}: Original issue price of the bond which does not change over time. 
        
        \item \textbf{Coupon Rate $r$}: Interest rate applied to the face value and is set by issuer (fixed). 
        
        \item \textbf{Price $P$}: What the investor pays for the bond.
        \item \textbf{Redemption Value of Bond $C$}: $ F = C $ until otherwise stated. $C$ is the amount the bondholder gets at maturity. 
        \item \textbf{Interest rate per payment period $i$}: Fluctuates based on market. 
        \item \textbf{Number of coupon payments $n$}
    \end{itemize}

\end{formula}

%-------------------------------------------
\section{Bond yield}

\begin{definition}
    \textbf{Bond Yield/Yield to Maturity (YTM)} is the rate of return $i$ you'll earn if you hold the bond to maturity, 
    assuming all coupons are reinvested at the same rate. It is expressed as percentage ($\%$). Components of YTM: 
    \begin{enumerate}
        \item Current bond price $P$
        \item Coupon payments $Fr$
        \item Time to maturity $n$
        \item Face Value/Par value $F$
    \end{enumerate}
\end{definition}

\begin{formula}
    \textbf{(Nominal Yield)} 
    \[
        \text{Nominal Yield} = \frac{\text{Annual Coupon}}{\text{Par Value}}
    \]  
\end{formula}

\begin{formula}
    \textbf{(Current Yield)} 
    \[
        \text{Current Yield} = \frac{\text{Annual Coupon}}{\text{Bond Price}}
    \]  
\end{formula}

\begin{formula}
    \textbf{(Yield to Maturity (YTM))} is the effective return based on what the bond costs today. 
\end{formula}

%---------------------------------------------
\section{Bond pricing formula}

\begin{formula} (Bond Price)
    For a bond with coupons, its price is 
    \[
        P(i) = \text{PV of Coupon payments} + \text{PV of Redemption value} = F \cdot r \cdot a_{\angl{n}i} + Cv^{n}
    \]
\end{formula}











%----------------------------------
\section{Premium/Discount}

\begin{comments}
    Bonds dont always trade at \textbf{par value} (i.e. $P$ is not always equal to $F$). This is because
    coupon rate $r$ is different from (market) yield rate/YTM $i$. In addition, price of bond $P$ is 
    dependent on market yield $i$. 
    \begin{itemize}
        \item If $r = i$ or $Fr = Ci$, bond is sold at par. 
        \item If $r > i$ or $Fr > Ci$, bond is sold at a premium. 
        \item If $r < i$ or $Fr < Ci$, bond is issed at a discount. 
    \end{itemize}
    Note that $F = C$ by default. 
\end{comments}


\begin{comments}
    If $r < i$, the bond is less attractive to investors. So the issuer must lower the price to make it appealing and the bond is sold at discount. 
    Similarly, if $r > i$, the issuer will raise the price as investors are willing to pay higher price for the bond.

    \begin{tabular}{|c|c|c|}
    \hline
    \textbf{Condition} & \textbf{Price Compared to \( C \)} & \textbf{Called} \\
    \hline
    \( r < i \) & \( P < C(=F) \) & \textbf{Discount bond} \\
    \( r = i \) & \( P = C(=F) \) & \textbf{Par} \\
    \( r > i \) & \( P > C(=F) \) & \textbf{Premium bond} \\
    \hline
    \end{tabular}

\end{comments}

\begin{formula}
    Price of a bond can be calculated using Premium/Discount: 
    \[
        P(i) = C + (F \cdot r - C \cdot i) \text{a}_{\angl{n}i}
    \]
    where $Fr - Ci$ is the premium or discount $\textbf{per period}$. They are cash flows over $n$ periods and 
    seen as an annuity of coupon payments. 
    \begin{itemize}
        \item \textbf{Premium}: $P-C = (Fr-Ci)a_{\angl{n}i}$
        \item \textbf{Discount}: $C-P = (Ci-Fr)a_{\angl{n}i}$
    \end{itemize}

    Note that $Fr$ is the coupon payment, and $Ci$ is the market interest payment. 
\end{formula}









%---------------------------------------------
\section{Bond duration}
\subsection{Par Bond Duration}

\begin{comments}
    When the bond is sold at par (i.e., when the yield $i=r$), its Macaulay Duration is: 
\begin{align*}
\text{MacD} &= \frac{F(r(Ia)_{\angl{n}} + nv^n)}{Fra_{\angl{n}} + Fv^{n}} \\
            &= \frac{r(Ia)_{\angl{n}} + nv^n}{ra_{\angl{n}} + v^{n}}  \\
            &= \frac{r(1+i)a_{\overline{n}|i} + (i - r)n v^n}{r + (i - r)v^n}
\end{align*}

Since $i=r$ (bond sold at par), the equation becomes
\[
    \text{MacD} = \frac{r(1+i)a_{\angl{n}i}}{r} = (1+i)a_{\angl{n}i} = \ddot{a}_{\angl{n}i}
\]
\end{comments}
\begin{formula}
    The Macaulay Duration of a par bond is 
    \[
    \text{MacD} = \ddot{a}_{\angl{n}i}
    \]
\end{formula}

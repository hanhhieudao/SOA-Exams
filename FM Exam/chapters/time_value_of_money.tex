\chapter{Interest Theory}
%-------------------------------------------------------
\section{Amount and Accumulation Functions}

\begin{definition}
    \textbf{Amount function} $A(t)$ refers to value of the investment at time t. 
\end{definition}

\begin{definition}
    \textbf{Accumulation function} $a(t)$ refers to value of \$1, which was invested at time 0, at time t. 
\end{definition}

\begin{formula}
    relationship between A(t) and a(t) is 
    \[
        A(t) = A(0) \cdot a(t)
    \]  
    where 
    \begin{itemize}
        \item A(0) is money you invested at time 0. 
        \item a(t)  tells how much \textbf{each dollar} grows to by time t. 
    \end{itemize}
\end{formula}











%-------------------------------------------------------
\section{Force of Interest}

\begin{definition}
    The \textbf{force of interest}, denoted by  $\delta$, measures how fast monet grows in a particular instant in time.
    By definition, the \textbf{force of interest} at time t, denoted by  $\delta(t)$, is: 
    \[
            \delta(t) = \frac{A'(t)}{A(t)}
    \]

    or it is also called as \textbf{instantaneous rate of growth} of the investment. 
\end{definition}

\subsection{Constant Force of Interest}
\begin{formula}
    When $\delta$ is constant, then $\delta(t)$ = $\delta$. 
    Continuous compounding is 
    \[
        A(t) = A(0) \cdot e^{\delta t}
    \]
\end{formula}

\begin{formula}
    Effective interest rate and force of interest:
    \[
        i = e^{\delta} - 1   
    \]

    and 
    \[
        \delta = \text{ln}(1+i)
    \]
\end{formula}

%-------------------------------
\section{Present value}

\subsection{Accumulation function with compound and simple interests}
\begin{definition}
    With \textbf{compound interest}, the accumulation function is 
    \[
        a(t) = (1+i)^t
    \]  
    Reminds that $a(t)$ tells how \$1 grows over t periods at (compound) interest rate $i$. 
\end{definition}

\begin{definition}
    With \textbf{simple interest}, the accumulation function is 
    \[
        a(t) = 1+it
    \]  
    Reminds that $a(t)$ tells how \$1 grows over t periods at (simple) interest rate $i$. 
\end{definition}



\subsection{Discounting}

\begin{comments}
    \begin{enumerate}
        \item \textbf{Discounting} What is \$1 in future worth today? $\rightarrow (1+i)^t$ after t periods
        \item \textbf{Accumulation} What does \$1 today grow to in future? $\rightarrow \frac{1}{(1+i)^t}$ after t periods
    \end{enumerate}
\end{comments}

\subsection{Discount factor}

\begin{definition}
    \textbf{Discount factor} converts future money into present value. For $t$ periods, 
    the discount factor is 
    \[
        v^t = \frac{1}{(1+i)^t}
    \]  

    Discount factor during nth period is

    \[
    (1+i_n)^{-1} = \frac{A(n-1)}{A(n)}
    \]
\end{definition}


\subsection{Present Value}

\begin{definition}
    Present Value with compound interest: 
    \[
        \text{PV} = \frac{\text{FV}}{(1+i)^{t}} = \text{FV} \cdot v^t
    \]  
\end{definition}
   

\begin{definition}
    Present Value with compound interest and cash flows: 
    \[
        \text{PV}_{t=0} = \sum_{k=1}^{n} C_k v(t_k)
    \]
\end{definition}

%---------------------------
\section{Future Value}
\begin{definition}
    With cash flows $C_1, C_2, \dots, C_n$ received at times $t_1, t_2, \dots, t_n$, then
    \[
        \text{FV}_{t=n} = \sum_{k=1}^{n} C_k a(t_k)
    \]
    \begin{itemize}
        \item $C_k$: cash flow received at time $t_k$
        \item $a(t_k)$: accumulation factor from time $t_k$ to the final time $t=n$
        \item $\text{FV}_{t=n}$: future value at time $t=n$
    \end{itemize}
\end{definition}